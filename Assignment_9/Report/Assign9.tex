
%%%%%%%%%%%%%%%%%%%%%%%%%%%%%%%%%%%%%%%%%
% Programming/Coding Assignment
% LaTeX Template
%
% This template has been downloaded from:
% http://www.latextemplates.com
%
% Original author:
% Ted Pavlic (http://www.tedpavlic.com)
%
% Note:
% The \lipsum[#] commands throughout this template generate dummy text
% to fill the template out. These commands should all be removed when 
% writing assignment content.
%
% This template uses a Perl script as an example snippet of code, most other
% languages are also usable. Configure them in the "CODE INCLUSION 
% CONFIGURATION" section.
%
%%%%%%%%%%%%%%%%%%%%%%%%%%%%%%%%%%%%%%%%%

%----------------------------------------------------------------------------------------
%	PACKAGES AND OTHER DOCUMENT CONFIGURATIONS
%----------------------------------------------------------------------------------------



\documentclass{article}

\usepackage{fancyhdr} % Required for custom headers
\usepackage{lastpage} % Required to determine the last page for the footer
\usepackage{extramarks} % Required for headers and footers
\usepackage[usenames,dvipsnames]{color} % Required for custom colors
\usepackage{graphicx} % Required to insert images
\usepackage{listings} % Required for insertion of code
\usepackage{courier} % Required for the courier font
\usepackage{lipsum} % Used for inserting dummy 'Lorem ipsum' text into the template
\usepackage{setspace}
\usepackage{color}
\usepackage{comment}
\usepackage{caption}
\usepackage[T1]{fontenc}
\usepackage{hyperref}
%\usepackage{natbib}
\usepackage{underscore}
\usepackage{subfigure}
\usepackage{fixltx2e}

\hypersetup{
    colorlinks=true,
    linkcolor=blue,
    filecolor=magenta,      
    urlcolor=cyan,
    breaklinks=true
}

\usepackage[]{algorithm2e}
\usepackage{pdfpages}
\usepackage{tikz}




%For python inclusion (http://widerin.org/blog/syntax-highlighting-for-python-scripts-in-latex-documents)
\definecolor{Code}{rgb}{0,0,0}
\definecolor{Decorators}{rgb}{0.5,0.5,0.5}
\definecolor{Numbers}{rgb}{0.5,0,0}
\definecolor{MatchingBrackets}{rgb}{0.25,0.5,0.5}
\definecolor{Keywords}{rgb}{0,0,1}
\definecolor{self}{rgb}{0,0,0}
\definecolor{Strings}{rgb}{0,0.63,0}
\definecolor{Comments}{rgb}{0,0.63,1}
\definecolor{Backquotes}{rgb}{0,0,0}
\definecolor{Classname}{rgb}{0,0,0}
\definecolor{FunctionName}{rgb}{0,0,0}
\definecolor{Operators}{rgb}{0,0,0}
\definecolor{Background}{rgb}{0.98,0.98,0.98}

% Margins
\topmargin=-0.45in
\evensidemargin=0in
\oddsidemargin=0in
\textwidth=6.5in
\textheight=9.0in
\headsep=0.25in

\linespread{1.1} % Line spacing

% Set up the header and footer
\pagestyle{fancy}
\lhead{\hmwkAuthorName} % Top left header
\chead{\hmwkClass\ (\hmwkClassInstructor\ \hmwkClassTime): \hmwkTitle} % Top center head
\chead{\hmwkClass\ (\hmwkClassInstructor): \hmwkTitle} % Top center head
\rhead{\firstxmark} % Top right header
\lfoot{\lastxmark} % Bottom left footer
\cfoot{} % Bottom center footer
\rfoot{Page\ \thepage\ of\ \protect\pageref{LastPage}} % Bottom right footer
\renewcommand\headrulewidth{0.4pt} % Size of the header rule
\renewcommand\footrulewidth{0.4pt} % Size of the footer rule

\setlength\parindent{0pt} % Removes all indentation from paragraphs

%----------------------------------------------------------------------------------------
%	CODE INCLUSION CONFIGURATION
%----------------------------------------------------------------------------------------

\definecolor{MyDarkGreen}{rgb}{0.0,0.4,0.0} % This is the color used for comments
\lstloadlanguages{Perl} % Load Perl syntax for listings, for a list of other languages supported see: ftp://ftp.tex.ac.uk/tex-archive/macros/latex/contrib/listings/listings.pdf
\lstset{language=Perl, % Use Perl in this example
        frame=single, % Single frame around code
        basicstyle=\small\ttfamily, % Use small true type font
        keywordstyle=[1]\color{Blue}\bf, % Perl functions bold and blue
        keywordstyle=[2]\color{Purple}, % Perl function arguments purple
        keywordstyle=[3]\color{Blue}\underbar, % Custom functions underlined and blue
        identifierstyle=, % Nothing special about identifiers                                         
        commentstyle=\usefont{T1}{pcr}{m}{sl}\color{MyDarkGreen}\small, % Comments small dark green courier font
        stringstyle=\color{Purple}, % Strings are purple
        showstringspaces=false, % Don't put marks in string spaces
        tabsize=5, % 5 spaces per tab
        %
        % Put standard Perl functions not included in the default language here
        morekeywords={rand},
        %
        % Put Perl function parameters here
        morekeywords=[2]{on, off, interp},
        %
        % Put user defined functions here
        morekeywords=[3]{test},
       	%
        morecomment=[l][\color{Blue}]{...}, % Line continuation (...) like blue comment
        numbers=left, % Line numbers on left
        firstnumber=1, % Line numbers start with line 1
        numberstyle=\tiny\color{Blue}, % Line numbers are blue and small
        stepnumber=5 % Line numbers go in steps of 5
}

% Creates a new command to include a perl script, the first parameter is the filename of the script (without .pl), the second parameter is the caption
\newcommand{\perlscript}[2]{
\begin{itemize}
\item[]\lstinputlisting[caption=#2,label=#1]{#1.pl}
\end{itemize}
}


%----------------------------------------------------------------------------------------
%	DOCUMENT STRUCTURE COMMANDS
%	Skip this unless you know what you're doing
%----------------------------------------------------------------------------------------

% Header and footer for when a page split occurs within a problem environment
\newcommand{\enterProblemHeader}[1]{
\nobreak\extramarks{#1}{#1 continued on next page\ldots}\nobreak
\nobreak\extramarks{#1 (continued)}{#1 continued on next page\ldots}\nobreak
}

% Header and footer for when a page split occurs between problem environments
\newcommand{\exitProblemHeader}[1]{
\nobreak\extramarks{#1 (continued)}{#1 continued on next page\ldots}\nobreak
\nobreak\extramarks{#1}{}\nobreak
}

\setcounter{secnumdepth}{0} % Removes default section numbers
\newcounter{homeworkProblemCounter} % Creates a counter to keep track of the number of problems

\newcommand{\homeworkProblemName}{}
\newenvironment{homeworkProblem}[1][Problem \arabic{homeworkProblemCounter}]{ % Makes a new environment called homeworkProblem which takes 1 argument (custom name) but the default is "Problem #"
\stepcounter{homeworkProblemCounter} % Increase counter for number of problems
\renewcommand{\homeworkProblemName}{#1} % Assign \homeworkProblemName the name of the problem
\section{\homeworkProblemName} % Make a section in the document with the custom problem count
\enterProblemHeader{\homeworkProblemName} % Header and footer within the environment
}{
\exitProblemHeader{\homeworkProblemName} % Header and footer after the environment
}

\newcommand{\problemAnswer}[1]{ % Defines the problem answer command with the content as the only argument
\noindent\framebox[\columnwidth][c]{\begin{minipage}{0.98\columnwidth}#1\end{minipage}} % Makes the box around the problem answer and puts the content inside
}

\newcommand{\homeworkSectionName}{}
\newenvironment{homeworkSection}[1]{ % New environment for sections within homework problems, takes 1 argument - the name of the section
\renewcommand{\homeworkSectionName}{#1} % Assign \homeworkSectionName to the name of the section from the environment argument
\subsection{\homeworkSectionName} % Make a subsection with the custom name of the subsection
\enterProblemHeader{\homeworkProblemName\ [\homeworkSectionName]} % Header and footer within the environment
}{
\enterProblemHeader{\homeworkProblemName} % Header and footer after the environment
}

%----------------------------------------------------------------------------------------
%	NAME AND CLASS SECTION
%----------------------------------------------------------------------------------------

\newcommand{\hmwkTitle}{Assignment\ \#9 } % Assignment title
%\newcommand{\hmwkDueDate}{Monday,\ January\ 1,\ 2012} % Due date
\newcommand{\hmwkClass}{Web Science} % Course/class
%\newcommand{\hmwkClassTime}{10:30am} % Class/lecture time
\newcommand{\hmwkClassInstructor}{Alexander Nwala} % Teacher/lecturer
\newcommand{\hmwkAuthorName}{Puneeth Bikkasandra} % Your name

%----------------------------------------------------------------------------------------
%	TITLE PAGE
%----------------------------------------------------------------------------------------

\title{
\vspace{2in}
\textmd{\textbf{\hmwkClass:\ \hmwkTitle}}\\
%\normalsize\vspace{0.1in}\small{Due\ on\ \hmwkDueDate}\\
%\vspace{0.1in}\large{\textit{\hmwkClassInstructor\ \hmwkClassTime}}
\vspace{0.1in}\large{\textit{\hmwkClassInstructor}}
\vspace{3in}
}

\author{\textbf{\hmwkAuthorName}}
\date{Tuesday, May 1, 2018} % Insert date here if you want it to appear below your name

%----------------------------------------------------------------------------------------

\begin{document}

\maketitle
\newpage



%----------------------------------------------------------------------------------------
%	TABLE OF CONTENTS
%----------------------------------------------------------------------------------------

%\setcounter{tocdepth}{1} % Uncomment this line if you don't want subsections listed in the ToC

\newpage
\tableofcontents
\newpage

%----------------------------------------------------------------------------------------
%	PROBLEM 1
%----------------------------------------------------------------------------------------

% To have just one problem per page, simply put a \clearpage after each problem

\begin{homeworkProblem}

Using the data from A7:
\begin{enumerate}
\item\textbf{}Consider each row in the blog-term matrix as a 1000 dimension vector, 
corresponding to a blog.  

\item\textbf{}Use knnestimate() to compute the nearest neighbors for both:

\begin{verbatim}
	http://f-measure.blogspot.com/
	http://ws-dl.blogspot.com/
\end{verbatim}

	for k={1,2,5,10,20}.

	Use cosine distance metric (chapter 8) not euclidean distance. 
	So you have to implement numpredict.cosine() instead of using 
	numpredict.euclidean() in:
	https://github.com/arthur-e/Programming-Collective-Intelligence/blob/master/chapter8/numpredict.py
	
\end{enumerate}
 
 
%\problemAnswer{

 \textbf{SOLUTION :}\\
 
I have solved the problem as described in the below steps :

  \begin{enumerate}
 	 \item\textbf{} I have used the "blogData.txt" blog metric from A7 and code files from\\ \textbf{Programming Collective Intelligence} text book.
  	\item\textbf{}  Modified the "numpredict" file to add the cosine similarity function
	 \item\textbf{}  Created a new file ,"nearestEstimate"
	 \item\textbf{} Passed the title of the two blogs, whose nearest neighbours need to be determined             			             					      
 \end{enumerate}
 
  \begin{lstlisting}[language=Python, caption=numpredict.py]
from random import random,randint
import math

def wineprice(rating,age):
  peak_age=rating-50
  
  # Calculate price based on rating
  price=rating/2
  if age>peak_age:
    # Past its peak, goes bad in 10 years
    price=price*(5-(age-peak_age)/2)
  else:
    # Increases to 5x original value as it
    # approaches its peak
    price=price*(5*((age+1)/peak_age))
  if price<0: price=0
  return price


def wineset1():
  rows=[]
  for i in range(300):
    # Create a random age and rating
    rating=random()*50+50
    age=random()*50

    # Get reference price
    price=wineprice(rating,age)
    
    # Add some noise
    price*=(random()*0.2+0.9)

    # Add to the dataset
    rows.append({'input':(rating,age),
                 'result':price})
  return rows

def euclidean(v1,v2):
  d=0.0
  for i in range(len(v1)):
    d+=(v1[i]-v2[i])**2
  return math.sqrt(d)


def cosineDistance(x, y):
    return np.dot(x, y) / (np.sqrt(np.dot(x, x)) * np.sqrt(np.dot(y, y)))


def getdistances(data,vec1):
  distancelist=[]
  
  # Loop over every item in the dataset
  for i in range(len(data)):
    vec2=data[i]
    
    # Add the distance and the index
    distancelist.append((cosineDistance(vec1,vec2),i))
  
  # Sort by distance
  distancelist.sort()
  return distancelist

def knnestimate(data,vec1,k=5):
  # Get sorted distances
  dlist=getdistances(data,vec1)
  neighbors = dlist[-k-1:-1]
  return neighbors

def inverseweight(dist,num=1.0,const=0.1):
  return num/(dist+const)

def subtractweight(dist,const=1.0):
  if dist>const: 
    return 0
  else: 
    return const-dist

def gaussian(dist,sigma=5.0):
  return math.e**(-dist**2/(2*sigma**2))

def weightedknn(data,vec1,k=5,weightf=gaussian):
  # Get distances
  dlist=getdistances(data,vec1)
  avg=0.0
  totalweight=0.0
  
  # Get weighted average
  for i in range(k):
    dist=dlist[i][0]
    idx=dlist[i][1]
    weight=weightf(dist)
    avg+=weight*data[idx]['result']
    totalweight+=weight
  if totalweight==0: return 0
  avg=avg/totalweight
  return avg

def dividedata(data,test=0.05):
  trainset=[]
  testset=[]
  for row in data:
    if random()<test:
      testset.append(row)
    else:
      trainset.append(row)
  return trainset,testset

def testalgorithm(algf,trainset,testset):
  error=0.0
  for row in testset:
    guess=algf(trainset,row['input'])
    error+=(row['result']-guess)**2
    #print row['result'],guess
  #print error/len(testset)
  return error/len(testset)

def crossvalidate(algf,data,trials=100,test=0.1):
  error=0.0
  for i in range(trials):
    trainset,testset=dividedata(data,test)
    error+=testalgorithm(algf,trainset,testset)
  return error/trials

def wineset2():
  rows=[]
  for i in range(300):
    rating=random()*50+50
    age=random()*50
    aisle=float(randint(1,20))
    bottlesize=[375.0,750.0,1500.0][randint(0,2)]
    price=wineprice(rating,age)
    price*=(bottlesize/750)
    price*=(random()*0.2+0.9)
    rows.append({'input':(rating,age,aisle,bottlesize),
                 'result':price})
  return rows

def rescale(data,scale):
  scaleddata=[]
  for row in data:
    scaled=[scale[i]*row['input'][i] for i in range(len(scale))]
    scaleddata.append({'input':scaled,'result':row['result']})
  return scaleddata

def createcostfunction(algf,data):
  def costf(scale):
    sdata=rescale(data,scale)
    return crossvalidate(algf,sdata,trials=20)
  return costf

weightdomain=[(0,10)]*4

def wineset3():
  rows=wineset1()
  for row in rows:
    if random()<0.5:
      # Wine was bought at a discount store
      row['result']*=0.6
  return rows

def probguess(data,vec1,low,high,k=5,weightf=gaussian):
  dlist=getdistances(data,vec1)
  nweight=0.0
  tweight=0.0
  
  for i in range(k):
    dist=dlist[i][0]
    idx=dlist[i][1]
    weight=weightf(dist)
    v=data[idx]['result']
    
    # Is this point in the range?
    if v>=low and v<=high:
      nweight+=weight
    tweight+=weight
  if tweight==0: return 0
  
  # The probability is the weights in the range
  # divided by all the weights
  return nweight/tweight

from pylab import *

def cumulativegraph(data,vec1,high,k=5,weightf=gaussian):
  t1=arange(0.0,high,0.1)
  cprob=array([probguess(data,vec1,0,v,k,weightf) for v in t1])
  plot(t1,cprob)
  show()


def probabilitygraph(data,vec1,high,k=5,weightf=gaussian,ss=5.0):
  # Make a range for the prices
  t1=arange(0.0,high,0.1)
  
  # Get the probabilities for the entire range
  probs=[probguess(data,vec1,v,v+0.1,k,weightf) for v in t1]
  
  # Smooth them by adding the gaussian of the nearby probabilites
  smoothed=[]
  for i in range(len(probs)):
    sv=0.0
    for j in range(0,len(probs)):
      dist=abs(i-j)*0.1
      weight=gaussian(dist,sigma=ss)
      sv+=weight*probs[j]
    smoothed.append(sv)
  smoothed=array(smoothed)
    
  plot(t1,smoothed)
  show()
\end{lstlisting}

Below code determines the nearest distance through cosine similarity :

 \begin{lstlisting}[language=Python, caption=nearestEstimate.py]
import clusters
import numpredict
def findNearestNeighbour(i, data, k):
    testing = data[i]
    neighbors = numpredict.knnestimate(data, testing, k)
    for i in neighbors:
        print(blogs[i])


blogs, text, data = clusters.readfile("blogDataForknn.txt")
for name in "F-Measure", "Web Science and Digital Libraries Research Group":
    for k in 1, 2, 5, 10, 20:
        print("Blog Name", name)
        print("For K", k)
        findNearestNeighbour(blogs.index(name), data, k)
        print("\n\n")
\end{lstlisting}

\newpage

 \begin{lstlisting}[caption=knn Result]
 ('Blog Name', 'F-Measure')
('For K', 1)
SPIN IT RECORDS Moncton 467A Main Street Moncton NB CANADA



('Blog Name', 'F-Measure')
('For K', 2)
Subterranean Noise
SPIN IT RECORDS Moncton 467A Main Street Moncton NB CANADA



('Blog Name', 'F-Measure')
('For K', 5)
PSI LAB
Some Call It Noise....
The Jeopardy of Contentment
Subterranean Noise
SPIN IT RECORDS Moncton 467A Main Street Moncton NB CANADA



('Blog Name', 'F-Measure')
('For K', 10)
Sonology
i'm in too truthful a mood
the fast break of champions
The Music Binge
A layer of chips
PSI LAB
Some Call It Noise....
The Jeopardy of Contentment
Subterranean Noise
SPIN IT RECORDS Moncton 467A Main Street Moncton NB CANADA



('Blog Name', 'F-Measure')
('For K', 20)
Steel City Rust
Captain Panda's Local & Independent Music Showcase
The Slow Music Movement
DaveCromwell Writes
Abu Everyday
The Run-Out Groove
Rants from the Pants
Broken Biscuit Records
My Life From A to Z
from a voice plantation
Sonology
i'm in too truthful a mood
the fast break of champions
The Music Binge
A layer of chips
PSI LAB
Some Call It Noise....
The Jeopardy of Contentment
Subterranean Noise
SPIN IT RECORDS Moncton 467A Main Street Moncton NB CANADA



('Blog Name', 'Web Science and Digital Libraries Research Group')
('For K', 1)
Subterranean Noise



('Blog Name', 'Web Science and Digital Libraries Research Group')
('For K', 2)
Sonology
Subterranean Noise



('Blog Name', 'Web Science and Digital Libraries Research Group')
('For K', 5)
Hip In Detroit
PSI LAB
macthemost
Sonology
Subterranean Noise



('Blog Name', 'Web Science and Digital Libraries Research Group')
('For K', 10)
She May Be Naked
ELLIA TOWNSEND A2
Diagnosis: No Radio
Abu Everyday
Rants from the Pants
Hip In Detroit
PSI LAB
macthemost
Sonology
Subterranean Noise



('Blog Name', 'Web Science and Digital Libraries Research Group')
('For K', 20)
DaveCromwell Writes
Playing Favorites
My Life From A to Z
ORGANMYTH
Nothing But Ordinary Glances At Extraordinary Things
Avidd Wallows' Blog
from a voice plantation
Mile In Mine
A2 MEDIA COURSEWORK JOINT BLOG
Alex Denney
She May Be Naked
ELLIA TOWNSEND A2
Diagnosis: No Radio
Abu Everyday
Rants from the Pants
Hip In Detroit
PSI LAB
macthemost
Sonology
Subterranean Noise
 \end{lstlisting}
%}
	
\end{homeworkProblem}
\clearpage
\newpage


\textbf{References}
\begin{enumerate}
\item\textbf{} https://cmry.github.io/notes/euclidean-v-cosine
\item\textbf{} http://scikit-learn.org/stable/modules/generated/sklearn.neighbors.DistanceMetric.html
\end{enumerate}
\end{document}
